Avec la méthode de déplacement développée au chapitre précédent, nous disposons de la principale <<brique>> élémentaire nécessaire à la construction d'un comportement intelligent. Pour ce faire, nous allons dans ce chapitre assembler ces briques à un niveau d'abstraction plus élevé, même s'il faudra encore développer quelques outils de plus bas niveau.

\section{Vue d'ensemble du comportement}

Après s'être initialisé, chaque footbot exécute a chaque pas une séquence d'opérations communes à tous les états possibles, suivie d'une séquence d'opérations propre à l'état courant.
\begin{lstlisting}[caption=Fonction step]
function step()
  local obstacleProximity, obstacleDirection, onSource,
        foundSource, backHome, gotSource,
        emerProx, emerDir
  obstacleProximity, obstacleDirection, onSource,
  foundSource, backHome, gotSource,
  emerProx, emerDir = doCommon()
  if emerProx>0 then
    emergencyAvoidance(emerProx, emerDir)
  elseif explore then
    doExplore(obstacleProximity, obstacleDirection,
              foundSource, gotSource)
  else
    doMine(obstacleProximity, obstacleDirection,
           onSource, backHome, foundSource)
  end
end
\end{lstlisting}

\subsection{Opérations Communes}

Les opérations communes à tous les états sont
\begin{itemize}
  \item L'écoute des capteurs <<vitaux>> c'est-à-dire qu'ils peuvent complètement écraser l'état courant: capteur de proximité (déclenche l'évitement d'urgence) et batterie (force le footbot à rentrer au nid)
  \item L'écoute des autres capteurs qui par essence doivent toujours être consultés: capteur de position, de couleur du sol (permet au footbot d'enregistrer des nouvelles sources à tout moment) et communication entre robots (idem).
  \item Lecture du capteur \emph{distance scanner (short range)} parce qu'il est utilisé en pratique quel que soit l'état du robot.
\end{itemize}

A part pour l'évitement d'urgence, toutes les décisions sont donc uniquement prises par les fonctions \emph{doMine} et \emph{doExplore}. La fonction \emph{doCommon} se contente de leur envoyer des signaux sous la forme des diverses variables renvoyées \emph{onSource, backHome,\ldots}.

L'implémentation détaillée de la fonction doCommon est donnée dans l'annexe \ref{appsec:doCommon}. L'évitement d'urgence est celui présenté dans la section \ref{sec:emerAvoid}, tandis que pour retourner le plus rapidement possible au nid, le footbot utilise simplement le déplacement présenté au chapitre \ref{chap:move} avec pour goal le centre du nid.

\subsection{Exploration\label{sec:explore}}

Tous les robots sont initialisés dans l'état \emph{explore}. Dans cet état, ils se déplacent selon une méthode d'exploration donnée jusqu'à recevoir le signal \emph{foundSource} ou \emph{gotSource}. Le premier signal signifie que le footbot a lui-même trouvé une nouvelle source, tandis que le second signal indique qu'une nouvelle source lui a été transmise. C'est la fonction \emph{doCommon} qui s'assure qu'une source potentielle est effectivement originale et la stocke alors dans la liste des sources connues. Cette liste est partagée avec les autres robots (voir section \ref{sec:commu}) et utilisée lorsque le robot est dans l'état \emph{exploite}.

Dans le cas où le robot a trouvé une source, il passe directement dans l'état exploite. Par contre, dans le cas où le robot reçu une nouvelle source, il décide aléatoirement s'il continue à explorer ou s'il se met à exploiter la ou les ressources qu'il connaît. La communication entre robot étant relativement efficace, ceci permet d'éviter que tous les robots se contentent d'une unique source dès qu'un seul individu en a fait la découverte. Une amélioration possible serait de permettre au robot de passer de l'état \emph{explore} à l'état \emph{exploite} même en l'absence de découverte originale s'il estime avoir déjà passé trop de temps à explorer.

Le parcours d'exploration pris par le robot est une composante importante et complexe de l'efficacité de la phase d'exploration. Des travaux précédents montrent cependant qu'une marche aléatoire (voir \cite{foraging} par exemple où la marche d'exploration est décrite en détail) peut suffire lorsque le nombre de robots est assez élevés, et leurs interactions suffisament développées. Nous nous limitons donc à un simple modèle de diffusion: les robots se déplacent en ligne droite et rebondissent sur tous les obstacles qu'ils rencontrent à la manière de simples particules. Ceci permet d'assurer que les robots migrent des zones les plus concentrées en footbots vers les zones les moins concentrées et que, plus globalement, les robots occupent tout l'espace disponible.

Le modèle utilisé est assez satisfaisant en théorie et en pratique mais n'est pas tout à fait adapté pour plusieurs raisons. Tout d'abord, la gestion de l'autonomie entraîne bien évidemment des retours fréquents des robots vers le nid, ce qui annule en partie l'effet désiré d'homogénéiser la concentration de robots. Ceci devrait être mieux pris en compte. D'autre part, ce modèle permet d'éviter relativement bien l'exploration d'une même zone par un nombre élevé de robots, mais ne permet pas de réellement privilégier d'éventuelles <<zones d'ombres>> qui n'ont encore jamais été explorées. Cependant, le principal facteur limitant semble bien être la portée extrèmement limitée des capteurs de couleur de sol.

L'implémentation du modèle de diffusion est détaillée en annexe \ref{appsec:gasLike}.

\subsection{Exploitation\label{sec:mine}}

Une fois passé dans l'état \emph{exploite}, le footbot fait simplement des allers-retours entre une des ressources connues et le nid en utilisant les méthodes de déplacement développées au chapitre \ref{chap:move}. Les opérations communes ont toujours lieu et le footbot continue donc à communiquer avec les autres robots et à lire la couleur du sol. Il reste donc à traiter du choix de la source à exploiter.

Ce choix est important: il est possible d'augmenter l'efficacité de l'exploitation en incitant les robots à exploiter les sources fournissant le meilleur rendement. Différentes méthodes peuvent être utilisés à cet effet, nous utilisons l'algorithme $\varepsilon$-greedy car il est simple, compréhensible, et utilise une seule variable qui représente directement le degré d'exploration~\cite{foraging}. Cet algorithme sélectionne avec une probabilité $1-\varepsilon$ la source au meilleur rendement et avec une probabilité $\varepsilon$ une source au hasard parmi les sources moins rentables.

\begin{algorithm}
\caption{Mise en place d'un score de qualité des sources}
\begin{algorithmic}
  \REQUIRE List of available ressources with their quality score
  \LOOP
    \STATE Pick ressource to mine using $\varepsilon$-greedy algorithm
    \STATE Go to ressource
    \STATE Update ressource quality score
    \STATE Go to nest
  \ENDLOOP
\end{algorithmic}
\end{algorithm}

Il est en fait plus sensé de mettre à jour l'indice de qualité non pas après un simple aller, mais après un aller-retour complet. Cependant, ceci aurait été plus difficile à mettre en pratique dans notre cas, et nous avons donc choisi la solution présentée ci-dessus. Le nouveau score de qualité est simplement donné par le nombre de trajets faits vers une source divisé par la quantité totale de batterie dépensée pour faire ces trajets.

L'utilisation d'une décision $\varepsilon$-greedy nécessitant déjà de trouver le meilleur indice de qualité, une amélioration possible serait de donner la possibilité au footbot de repasser dans l'état \emph{explore} si le meilleur score dans la liste passe en dessous d'une certaine valeur critique. Cependant, il faudrait alors aussi permettre au footbot de changer d'état dans l'autre sens à d'autres occasions que seulement s'il trouve ou reçoit une nouvelle source, comme il a été évoqué dans la section \ref{sec:explore}.

L'implémentation détaillée de l'évaluation et du choix des ressources est présentée dans l'annexe \ref{appsec:evalSource}.

\section{Derniers outils bas niveau}

\subsection{Communication\label{sec:commu}}

Vu la structure en boucle demandée par ARGoS, et grâce au pas d'éxécution assez court, les footbots peuvent entièrement partager leur connaissance des ressources sans que la limite de dix Bytes imposée par le système \emph{range and bearing} ne pose problème.

\begin{algorithm}
  \caption{Partage total de la connaissance des ressources}
  \begin{algorithmic}
  \REQUIRE Known ressources list
  \ENSURE Total sharing of the ressource knowledge
  \LOOP
  \STATE broadcast a different source
  \ENDLOOP
  \end{algorithmic}
\end{algorithm}

Dans notre projet, les ressources étant stockées sous forme de liste, les footbots diffusent à chaque pas la i\ieme{} source dans la liste où i est le numéro du pas courant modulo la longueur de la liste des ressources connues.

Une source étant représentée par un couple de nombre à virgule flottante compris entre -500 et 500, la conversion suivante est utilisée pour transmettre l'information:

\vspace{1em}
\makebox[\textwidth][c]{%
\noindent\begin{tabular}{ |c|c|c|c|c|c|c| }
\hline
Byte 1 & Byte 2 & Byte 3 & Byte 4 & Byte 5 & Byte 6 & Byte 7 \\ \hline
\multirow{3}{*}{type message} & \multirow{2}{*}{Signe x} & \multirow{2}{*}{Signe y} & Centaines x & Centaines y & Reste x & Reste y \\ \cline{4-7}
&&&$\left\lfloor \frac{\lvert x \rvert}{100} \right\rfloor$ & $\left\lfloor \frac{\lvert y \rvert}{100} \right\rfloor$
& $\scriptstyle\mathrm{round}(\lvert x \rvert \bmod 100)$ & $\scriptstyle\mathrm{round}(\lvert x \rvert \bmod 100)$ \\
\cline{2-7}
&\multicolumn{6}{ |c| }{coordonnées} \\
\hline
\end{tabular}
}

\vspace{1em}

Le premier Byte indique comment décoder et interpréter le reste du message. La table ci-dessus n'est donc valide que pour les messages dont les 6 Bytes après le type sont alloués à une position dans l'arène. Dans notre projet, les seuls messages échangés sont de ce type et transmettent la position d'une nouvelle source, mais cette manière de faire est flexible et permet de facilement rajouter d'autres échanges: il suffirait par exemple de changer le type en gardant le même encodage de position pour implémenter la partie communication du ravitaillement entre robots.

Lorsqu'un robot reçoit un message, il le décode en se servant du type du message et de l'encodage ci-dessus, et, dans le cas d'une nouvelle source, vérifie son originalité et la stocke alors en mémoire. Si le robot est déjà en train d'exploiter des ressources, le traitement s'arrête là. Sinon, le footbot doit alors choisir s'il commence à exploiter les ressources ou non, comme décrit dans la section \ref{sec:explore}.

L'implémentation en Lua de la communication est détaillée dans l'annexe \ref{appsec:commu}.

\subsection{Gestion de l'autonomie}

L'autonomie limitée des footbots impose aux footbots de revenir au nid à intervalles réguliers. Ceci est immédiat à implémenter, à l'exception de la problématique principale: quel indicateur utiliser pour effectuer la décision de rentrer au nid.

TODOOOOOOOOOOOOOOO

\section{Retour sur le comportement général\label{sec:tree}}

Pour conclure ce chapitre, reprenons un point de vue plus large sur le comportement que nous avons construit. La figure \ref{fig:genFlowchart} résument les différents états et transitions traversés par chaque footbot tout au long d'une expérience. Les transitions en pointillés représentent les deux axes d'amélioration que nous avons évoqués aux paragraphes \ref{sec:explore} et \ref{sec:mine}.
\begin{figure}[htb]
  \begin{tikzpicture}[node distance = 2cm,>=stealth, auto]
    % Place nodes
    \node [cloud] (init) {Start};
    \node [block, right of=init, xshift=0.7cm] (explore) {Explore};
    \node [block, below of=explore, yshift=-1cm] (store1) {Store and share it};
    \node [block, right of=explore, xshift=4cm] (store2) {Store and share it};
    \node [cloud, below of=store1] (mine) {Mine source(s)};
    \node [decision, below of=store2] (decide) {Start to Mine?};
    \node [block, below of=mine, yshift=-0.7cm] (store3) {Store and share it};
    % Draw edges
     \path [line] (init) -- (explore);
     \path [line] (explore) -- node [text width=2cm] {New source found}(store1);
     \path [line] (explore) -- node {New source received}(store2);
     \path [line] (store1) -- (mine);
     \path [line] (store2) -- (decide);
     \path [line] (decide) -- ++(2cm,0) node[xshift=.5cm, yshift=.7cm] {No} --++(0,4.5cm) -| (explore);
     \path [line] (decide) -- ++(0,-2cm) node [anchor=west]{Yes} |-(mine);
     \path [line] (mine) -- node[text width=3cm] {New source found or received}(store3);
     \path [line] (store3) -- ++(-3cm,0) |- (mine);
     \draw [dashed, <->](mine) .. controls ++(-2.5cm,3cm) .. node[yshift=2cm,xshift=1.8cm,text width=2.2cm,style={rectangle,fill=white}]{No satisfying source} node[yshift=-1.1cm,xshift=2cm,text width=2cm,style={rectangle,fill=white}]{No recent discovery}(explore);
\end{tikzpicture}
\caption{Comportement haut niveau de chaque robot~\cite{foraging}.\label{fig:genFlowchart}}
\end{figure}

Il s'agit maintenant de mettre ce comportement à l'épreuve, d'évaluer ses performances, d'examiner quelles possibilités d'optimisation et d'adaptation sont déjà offertes et d'identifier les défauts plus fondamentaux qui sonts présents.

\begin{subappendices}
  \appsec{Implémentation des opérations communes en Lua\label{appsec:doCommon}}
  \subsection{Appels faits par \emph{doCommon}}

  L'implémentation des fonctions appelées est donnée dans les sections traitant de ce que font ces fonctions (fonction \emph{listen} $\rightarrow$ communication, fonction \emph{closestObstacleDirection} $\rightarrow$ évitement d'obstacles, \ldots) à l'exception de \emph{checkGoalReached}, qui détecte de nouvelles sources et qui est présentée ci-dessous.
\begin{lstlisting}[caption=fonction doCommon]
function doCommon()
  local obstacleProximity, obstacleDirection,
        onSource, foundSource, backHome, gotSource,
        emerProx, emerDir

  odometry()
  onSource, foundSource, backHome = checkGoalReached()
  if #ressources>=1 then
    broadcastSource(chooseSourceToBroadcast())
  end
  gotSource = listen()
  shortObstaclesTable = updateObstaclesTable("short_range",
                                             shortObstaclesTable)
  obstacleProximity, obstacleDirection
  =closestObstacleDirection(shortObstaclesTable)

  battery=battery-BATT_BY_STEP
  backForBattery = backForBattery
                   or battery-batterySecurity
                              *BATT_BY_STEP
                              *math.sqrt(posX^2+posY^2)
                              /BASE_SPEED<10

  emerProx, emerDir=readProxSensor()
  if battery==0 then
    BASE_SPEED=0
    logerr("batt empty")
  end
  return obstacleProximity, obstacleDirection,
         onSource, foundSource, backHome,
         gotSource, emerProx, emerDir
end
\end{lstlisting}

\subsection{Détection de nouvelle sources et du nid}

\begin{lstlisting}[caption=fonction checkGoalReached]
function checkGoalReached()
   local foundSource, onSource, backHome=false,false,false
   local insideBlack, seeBlack = floorIsBlack()
   if seeBlack and math.sqrt((posX)^2+(posY)^2)>=90 then
      if insideBlack and sourceIsOriginal(posX,posY, ressources) then
         ressources[#ressources+1]={math.floor(posX),math.floor(posY),
                                      score=1, travels=0,bSpent=0}
         --store source. Initial score=1 which is way above average
         --(Forces footbots to try new sources out)
         foundSource=true
      end
      onSource=true
   elseif seeBlack and math.sqrt((posX)^2+(posY)^2)<=70 then
      if goalX==0 and goalY==0 then
         backHome=true
      end
      if backForBattery then
         batterySecurity=updateBattCoeff(battery,batterySecurity)
         backForBattery=false
      end
      battery=100
   end
   return onSource, foundSource, backHome
end
\end{lstlisting}

\begin{lstlisting}[caption=Détection de la couleur du sol]
function floorIsBlack()
  local insideBlack = true
  --will tell if robot is fully inside a source or the nest
  local seeBlack = false
  --will tell if robot is touching a source or the nest
  local clr,i
  for i=1,8 do
    clr = robot.base_ground[i].value
    seeBlack = seeBlack or clr== 0
    insideBlack = insideBlack and clr == 0
  end
  return insideBlack, seeBlack
end
\end{lstlisting}


\begin{lstlisting}[caption=Vérifier l'originalité de la source]
function sourceIsOriginal(x, y, rsc)
  local i=1
  local orgn=true
  while i<=#ressources and orgn do
    orgn=(math.sqrt((rsc[i][1]-x)^2 + (rsc[i][2]-y)^2)>ORGN_SRC_DST)
    --just check if it's far enough from sources you already know
    i=i+1
  end
  return orgn
end
\end{lstlisting}

\appsec{Implémentation du modèle de diffusion en Lua\label{appsec:gasLike}}

\begin{lstlisting}[caption=fonction doExplore]
function doExplore(obstacleProximity, obstacleDirection,
                    foundSource, gotSource)
  if foundSource then
    explore=false
    goalX,goalY=0,0
  end
  if gotSource then
    if robot.random.uniform()<MINE_PROB_WHEN_SRC_RECVD then
    --choose whether or not you start mining
      explore,goalX,goalY=false,0,0
      --refill battery before starting to mine
    end
  end
  if not backForBattery then
    gasLike(obstacleProximity, obstacleDirection)
  else
    obstaclesTable = updateObstaclesTable("long_range",obstaclesTable)
    move(obstaclesTable, obstacleProximity, obstacleDirection,0,0)
  end
end
\end{lstlisting}


\begin{lstlisting}[caption=Déplacement \emph{gaslike}, label=list:gaslike]
function gasLike(obstacleProximity, obstacleDirection)
  local goalAngle
  if obstacleProximity < 30
     and not(obstacleDirection<-PI/2 or obstacleDirection>PI/2)
     and not wasHit then
    wasHit = true --/!\Not local: allows footbot to remember
                  --if it's already trying to reach an angle
    newDirection = alpha+rebound(alpha,obstacleDirection)
    newDirection=setCoupure(newDirection)   --sets angle in [-PI,PI]
  end
  if wasHit then
    if abs(alpha-newDirection)<0.2 then --checks if angle is reached
      wasHit=false
    else
      goalAngle=newDirection-alpha
      goalAngle=setCoupure(goalAngle) --sets angle in [-PI,PI]
      getToGoal(goalAngle, EXPL_CONV)
      --EXPL_CONV is really high because we want to do sharp turns
    end
  else
    robot.wheels.set_velocity(BASE_SPEED,BASE_SPEED)
  end
end
\end{lstlisting}

\begin{lstlisting}[caption=Calcul de l'angle "réfléchi"]
function rebound(alpha, obstacleDirection)
  if obstacleDirection<=12 then   --obstacle is to the left
    newAngle = -2*(PI/2-obstacleDirection)
  else
    newAngle = 2*(PI/2-obstacleDirection)
  end
  newAngle=setCoupure(newAngle) --sets angle in [-PI,PI]
  return newAngle
end
\end{lstlisting}

\appsec[Implémentation du choix de ressource en Lua]{Implémentation du choix de ressource à exploiter en Lua\label{appsec:evalSource}}

\begin{lstlisting}[caption=fonction doMine]
function doMine(obstacleProximity, obstacleDirection, onSource, backHome)
   if onSource then
      if not hasMined then
         ressources[sourceId].travels=ressources[sourceId].travels +1
         travels=travels+1
         evalSource(sourceId, battery)
         goalX,goalY=0,0
      end
      hasMined=true
   elseif backHome then
      if hasMined then
         hasMined=false
      end
      sourceId,goalX,goalY=chooseNewSource(ressources)
   elseif backForBattery then
      if goalX~=0 and goalY~=0 then
         goalX,goalY=0,0
         evalSource(sourceId, 0)
      end
   end
   obstaclesTable = updateObstaclesTable("long_range",obstaclesTable)
   move(obstaclesTable, obstacleProximity, obstacleDirection,goalX,goalY)
end
\end{lstlisting}

\begin{lstlisting}[caption=Choix de la ressource à exploiter]
function chooseNewSource(rsc)
   local newSourceId
   if #rsc>1 then
      placeMaxAtOne(rsc)
      local pickBest=robot.random.uniform()
      if pickBest<(1-EPSILONGREED) then
         newSourceId=1
      else
         newSourceId=robot.random.uniform_int(2,#rsc+1)
      end
   else
      newSourceId=1
   end
   local x=rsc[newSourceId][1]
   local y=rsc[newSourceId][2]
   return newSourceId, x, y
end
\end{lstlisting}

\begin{lstlisting}[caption=Evaluation de la qualité de la source]
function evalSource(sourceId, battery)
   ressources[sourceId].bSpent=ressources[sourceId].bSpent+(100-battery)
   ressources[sourceId].score=ressources[sourceId].travels
                              /ressources[sourceId].bSpent
end
\end{lstlisting}

\appsec{Implémentation en Lua de la communication\label{appsec:commu}}

\subsection{Envoi de message}

\begin{lstlisting}[caption=Envoi d'un message différent à chaque pas]
function doCommon()
  ...
  if #ressources>=1 then
    broadcastSource(chooseSourceToBroadcast())
  end
  ...
end

function broadcastSource(x,y)
  local msg=sourceIn(x,y)
  robot.range_and_bearing.set_data(msg)
end

function chooseSourceToBroadcast()
  local i=(currentStep%#ressources)+1
  return ressources[i][1],ressources[i][2]
end
\end{lstlisting}

\begin{lstlisting}[caption=Encodage du message]
function sourceIn(x,y)
  local msgOut={1,sgnIn(x),sgnIn(y),
                math.floor(abs(x)/100),math.floor(abs(y)/100),
                math.floor(abs(x)%100),math.floor(abs(y)%100),
                0,0,0}
  return msgOut
end

function sgnIn(n)
   local sgn
   if n==0 then
      sgn=0
   else
      if n==abs(n) then
         sgn=1
      else
         sgn=2
      end
   end
   return sgn
end
\end{lstlisting}


\subsection{Réception de message}

\begin{lstlisting}[caption=Ecoute du capteur à chaque pas]
function doCommon()
  ...
  gotSource = listen()
  ...
end

function listen()
  local gotSource=false
  for i=1,#robot.range_and_bearing do
    if robot.range_and_bearing[i].data[1]==1 then
      local source = sourceOut(robot.range_and_bearing[i].data)
      if sourceIsOriginal(source[1],source[2],ressources) then
      --Check for originality
        source.score=1
        source.travels=0
        source.bSpent=0
        ressources[#ressources+1]=source
        gotSource=true
      end
    end
  end
  return gotSource
end
\end{lstlisting}

\begin{lstlisting}[caption=Décodage du message]
function sourceOut(msg)
   local x,y
   x=100*msg[4]+msg[6]
   if msg[2]==2 then
      x=-x
   end
   y=100*msg[5]+msg[7]
   if msg[3]==2 then
      y=-y
   end
   return {x,y}
end
\end{lstlisting}

\end{subappendices}
