The goal of this project was to design a swarm-intelligent behaviour for virtual robots using the ARGoS software. These robots had to explore an unknown environment with obstacles in order to ultimately loop between spots marked on the ground and a starting area. The fundamental paradigm was that a single set of rules would be followed independently by every robot, which would allow a swarm-intelligent behaviour to emerge through the robot interactions prescribed by these rules. For that to work, exploration, shortest path-finding and obstacle-avoidance algorithms were needed, along with elementary automated decision making, communication and odometry. These concepts were implemented using the ARGoS loop approach, which means that the same sequence of actions takes place at every step, while only events occurring during that step can influence these actions. However, a Dijkstra path-finding algorithm, which would only execute once before every trip while always providing the shortest trajectory, was also considered. A first working solution was produced using the Lua language, then put to the test and could quickly be enhanced accordingly thanks to the flexible framework. This allowed robots to loop between a starting area and spots of known position while avoiding collisions, and information was gathered on parameters meaningful for the experiment. Future development should be focused on optimizing these parameters and enabling robots to explore a fully unknown environment.