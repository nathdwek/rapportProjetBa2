\section{Général}

Le groupe est composé de 5 membres. Chaque membre a sa manière de travailler, de comprendre, de communiquer. Une des difficultés d'un travail d'équipe est de pouvoir combiner tous ces caractères pour que le projet se déroule dans les meilleures conditions et que chacun puisse trouver sa place. Donc pour comprendre le fonctionnement de chacun, chaque membre a dû présenter ses points forts et ses points faibles. Le but de cette démarche est de mettre en avant ses points forts et d'améliorer ses points faibles durant le projet. Des tensions sont tout de même survenues. En effet, le manque de communication a souvent mené de groupe à mal se coordonner.

\section{Organisation}

Le groupe se fixe une réunion par semaine. Les tâches à effectuer pour la prochaine réunion sont distribuées à la fin de celle-ci et notées dans les PV qui sont généralement envoyés dans les 48h. De rôles d'animateur et secrétaire sont réattribués toutes les 4 semaines. Pour diriger au mieux le projet, un diagramme de Gantt (cf.: annexe) a été conçu pour avoir une vision globale de l'avancé de celui-ci et également se fixer des dates butoirs.

\section{Communication}

Dans un premier temps, un groupe sur facebook a été créé. C'est un moyen simple et rapide de se communiquer les informations mais une communication par mail reste préférable. Pour le partage, des codes, des fichier et des sources des recherches, les dispositifs github, dropbox et zotero ont été mis en place. L'emploie de ces 3 dispositifs n'a pas encore été exploité au mieux mais son apprentissage est bénéfique non seulement dans le cadre de ce projet mais également pour plus tard. 