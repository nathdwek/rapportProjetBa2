Le but de ce projet était de créer une intelligence en essaim pour des robots modélisés dans le simulateur ARGoS. Ceux-ci devaient explorer un environnement inconnu qui comportait des obstacles afin d'ensuite faire des allers-retours entre des zones marquées aux sols et leur nid de départ. Le principe de base était que le même ensemble de règles devrait être suivi de manière indépendante par chaque robot; la caractéristique intelligente de l'ensemble de robots devant émerger à travers les interactions entre robots prescrites par ces règles. Pour cela, des procédures d'exploration, de recherche du plus court chemin et d'évitement furent nécessaires ainsi que des principes de base de prise de décision, d'odométrie et de communication. Ces concepts furent mis en pratique en utilisant l'approche loop d'ARGoS, qui implique que le robot exécute la même séquence d'opérations à chaque pas. Cependant, une recherche du plus court chemin Dijkstra, qui ne serait faite qu'une seule fois avant chaque trajet d'un robot, tout en trouvant toujours le chemin le plus court, fut aussi considéré. Une première solution fonctionnelle en Lua fut construite, testée et ensuite améliorée en conséquence grâce au turnaround loop très court offert par ARGoS. Cette solution permet aujourd'hui au robot de faire des allers-retours entre leur nid et une ou plusieurs ressources dont ils connaissent la position à l'avance, tout en évitant les collisions dans un environnement inconnu. De plus, des informations ont été collectées sur des paramètres influençant l'expérience. La deuxième partie du quadrimestre devrait être consacrée à l'optimisation de cette solution et à permettre au robot d'explorer un environnement afin de trouver la position des sources à exploiter. 