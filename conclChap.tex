En conclusion, une compréhension de l'intelligence artificielle, du fonctionnement des robots et l'apprentissage de Lua ont permis d'atteindre les objectifs du 1er quadrimestre, ceux-ci étant de guidé les robots du nid à la source et de les programmer pour qu'ils puissent y faire des allers-retours. 
Les prochains objectifs à atteindre sont la mise en place d'un moyen de communication et d'exploration optimale l'environnement. Quelques options vont probablement être ajoutées dont l'introduction d'un système de priorité, dans le cas ou 2 robots cherchent à s'éviter, et une batterie se déchargeant de manière différente suivant l'état du robot (au repos ou en mouvement). De plus, une étude du comportement en essaim généré par notre code sera menée afin d'évaluer l'«intelligence de l'essaim».