Le but du projet est de doter un essaim de robots d'un comportement intelligent afin qu'ils soient capables \cite{cahierCharges}
\begin{itemize}
  \item d'explorer un environnement inconnu afin de localiser des <<sources>> symbolisées par des taches noires au sol
  \item d'<<exploiter>> les sources découvertes en faisant des allers-retours entre celles-ci et le nid
  \item de partager l'information accumulée afin d'optimiser l'exploitation des sources
  \item d'accomplir ces tâches dans un environnement comportant des obstacles, tout en gérant leur autonomie limitée
\end{itemize}
L'essaim de robots et son environnement sont simulés par ARGoS, un simulateur développé par le laboratoire IRIDIA.

\section{Intérêt du projet}

Comme son nom l'indique, la robotique en essaim met en oeuvre un nombre élevé de robots afin d'effectuer une tâche. Ceci est très différent de ce qui se fait habituellement en robotique <<classique>> où un petit nombre de robots extrêmement sophistiqués est déployé afin de résoudre une problématique. Les principes fondamentaux derrière la programmation d'un essaim de robots peu coûteux mais au capacités plus limitées sont donc aussi différents: chaque individu ne doit plus être considéré comme infaillible, et la perte d'un robot prend moins d'importance, tant qu'elle profite à l'essaim tout entier. Ceci ouvre de nouvelles voies dans de nombreux domaines, par exemple dans le cas très concret du déminage ou lorsqu'il faut opérer dans une zone hautement hostile au sens plus général (intervention en milieu radioactif, en grande profondeur, \ldots).~\cite{swarmMini}

\section{Résultats attendus}

L'objectif premier est de développer un comportement qui permet aux robots de survivre et d'exploiter une ressource de manière autonome dans un environnement non connu à l'avance.

Dans un premier temps, les robots seront considérés comme omniscients et connaîtront donc l'environnement à explorer. Cette connaissance leur sera ensuite retirée. Une communication entre les robots pourra être envisagée par la suite et permettra nottament de mieux gérer l'information incomplète.

Même si la tâche à accomplir est au niveau de l'essaim, ce dernier ne sera jamais programmé directement. Le principe même du projet est de développer un comportement qui sera suivi par chaque robot indépendamment. Des interactions locales entre robots, physiques ou non, émergera un comportement global qui devra être étudié et être rendu prévisible.

Des outils de mesure, afin de calculer la qualité du comportement en essaim, devront être élaborés. Ils devront établir la performance des solutions proposées en fonction des objectifs initiaux.~\cite{cahierCharges}

\vspace{1em}Ce rapport a été divisé en quatre parties: une introduction à la notion d'intelligence artificielle et aux outils utilisés, la problématique du déplacement des robots, et enfin la mise en place à un niveau plus haut d'un comportement <<intelligent>> permettant de remplir les objectifs décrits plus haut. Pour finir, nous conclurons par l'évaluation des performances et les perspectives d'amélioration.
