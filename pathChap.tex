\section{Algorithmes déterministes classiques}

Dans le cas où les footbots disposent réellement d'une description complète de leur environnement, il est envisageable d'utiliser les algorithmes classiques de recherche du plus court chemin comme A* et Dijkstra. Ceux-ci fonctionnent en représentant l'environnement sous forme d'un graphe dont les n\oe{}uds représentent les positions accessibles et dont les branches associent à deux n\oe{}uds joignables le coût (ou la distance) qui les sépare. Ces algorithmes associent à un chemin en cours d'évaluation un coût total et sélectionnent au final le chemin au coût le moins élevé.

Dijkstra évalue le coût du chemin courant en calculant simplement la somme des coûts depuis le n\oe{}ud initial. Pour cette raison, il ne privilégie a priori aucune direction d'exploration et son exécution ne se termine que si tous les n\oe{}uds accessibles ont été évalués ou si le n\oe{}ud but a été <<visité>>\footnote{Dans le cas de Dijkstra, <<visité>> signifie que tous les n\oe{}uds adjacents ont été évalués.}.

A* rajoute au coût <<connu>> du chemin reliant l'origine au n\oe{}ud courant l'évaluation heuristique du coût du chemin reliant le point courant au but. Ceci permet de guider la recherche et de l'arrêter plus tôt si on a confiance en l'heuristique utilisée.~\cite{mehlhorn_shortest_2008,genetic2007}

Les algorithmes classiques de recherche du chemin ne sont pas immédiatement adaptables à un environnement partiellement connu ou en cours d'exploration. Des alternatives existent cependant, parmi lesquelles l'utilisation de l'intelligence en essaim, qui s'inspire en partie  des comportements originaux et très efficaces déjà présents dans la nature.

\section{Algorithme des fourmis\label{sec:ants}}

Cet algorithme est basé sur le comportement de certaines colonies de fourmis qui communiquent indirectement par dépôts de phéromones. Il se présente ainsi: lorsqu'une fourmi trouve une source de nourriture, elle dépose, lors de son retour vers la fourmilière, des phéromones tout au long du chemin qu'elle emprunte. Les autres fourmis suivent cette piste de phéromones pour exploiter la source de nourriture et déposent elles aussi des phéromones sur le chemin du retour, qui peut parfois être différent de la piste suivie initialement.

Comme les fourmis suivent préférentiellement les pistes comportant le plus de phéromones, et puisque les phéromones s'évaporent après un certain laps de temps, le chemin le plus court devient rapidement la piste dominante, car plus de fourmis le traversent et déposent des phéromones par unité de temps. De plus, comme les autres pistes sont alors délaissées, cet algorithme converge très fortement car le chemin le plus court devient rapidement l'unique chemin marqué.~\cite{antOpti}

Cette approche a été utilisée par \cite{pheromonesForaging}, ce qui montre qu'il est possible d'appliquer cet algorithme aux footbots dans ARGoS. Ici, ce sont des messages transmis de robot à robot qui jouent le rôle des phéromones, et la rapidité avec laquelle un seule message traverse toute une chaîne de robot qui indique l'<<intensité>> d'une piste; mais les principes fondamentaux restent les mêmes.

\vspace{1em}
Par manque de temps, nous n'avons pas pu implémenter une recherche du chemin efficace en non-omniscient comme ci-dessus, et nous avons choisi de ne pas perdre de temps à implémenter une recherche ``classique'' car elle serait inutile en non-omniscient, qui est le but final du projet.
